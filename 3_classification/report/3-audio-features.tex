
\section{Audio Features}
\label{sec:Audio Features}




\subsection{Which subset of audio features did you select for your final classifier? Describe the selection process and the criteria you used to make your choice.}
\label{sec:Audio Features:a}

After reflecting on the results from section \hyperref[sec:Labeling Function:b]{1.), b.)}, we have decided to run our experiments only on the 'embeddings' feature, as there was also was no significant drop in performance for tests we ran on a small scale. Our reason for only using that feature stays the same as discussed in the last paragraph of that section: Seemingly heavy correlations and possible redundancy of the other three features considering the possible creation process of these audio embeddings.


\subsection{Did you apply any preprocessing to the audio features? If so, explain which techniques you used and why they were necessary.}
\label{sec:Audio Features:b}

Preprocessing is an important step for ensuring consistency in the data and making the lives of models like Support-Vector-Machines much easier as the features are confined in a smaller space. We used the global mean and standard deviation from the training data and normalized our training, validation and test sets using these global statistics (to avoid data leakage). We then additionally normalized the features frame-wise to have unit mean and variance for each example.




