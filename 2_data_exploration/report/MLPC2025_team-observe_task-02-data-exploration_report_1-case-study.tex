
\section{Case Study}
\label{sec:Case Study}

To find 2 interesting records that were edited by multiple annotators was the challenge here. 
%An important assumption here is the correctness and accuracy of the titles and keywords. \\
%We first looked in ``metadata.csv'' to see which files had more than one annotator. 
%This resulted in a list of 149 files. 
%We then looked at the ``metadata\_title\_embeddings.npz'' and the ``metadata\_keywords\_embeddings.npz'' in order 
%to be able to draw some conclusions. \\
%At the same time, we also checked the standardization of the embedding. \\
%The approach was that files with very clear embedding (target: 1.0 in one place) lead to very clear annotations. 
%However, this did not produce any useful results. \\
%The next approach 
We used the file ``annotations.csv'' and the corresponding ``annotations\_text\_embeddings.npz''. 
All annotations with more than one annotator were searched for. 
%This resulted in a list of 731 annotations and 1468 annotations in total. There were 6 files with 3 annotations each. 
%From the associated text embeddings, we compared all embeddings of an annotator and an annotation with all other annotators by distance and 
%subsequently take the following 2 files out. \\
We compared all embeddings of an annotator and an annotation with all other annotators by distance.
The file with the largest difference (largest distance) is '568273.mp3' and the file with the highest similarity (smallest distance) is '203149.mp3'.

\subsection{Identify similarities or differences between temporal and textual annotations from different annotators.}
\label{sec:Case Study:a}

The temporal windows fit together very well. The keywords in the different describe two different things (violin, drone).
This file shows that our assumption of the correctness and accuracy of the titles and keywords is wrong. 
The word drone appears in the title and the keywords (Table~\ref{tab:largest difference}).

\begin{table}[h]
  \caption{File with the largest difference}
  \label{tab:largest difference}
  \centering
  \begin{tabular}{cclrrp{7.5cm}}
    \toprule
    index & annotator & filename & onset & offset & text \\
    \midrule
    7323 & 1 & 568273.mp3 & 0.0 & 20.073243 & A sharp, loud violin plays rapidly at a concert. \\
    27703 & 2 & 568273.mp3 & 0.0 & 20.028386 & A sustained ambient drone with granular and spectral textures. \\
    \bottomrule
  \end{tabular}
\end{table}

The collected temporal windows match very well. The keywords in the similar files describe two identical things (cows, birds).
Annotator 1 has only worked in much more detail (Table~\ref{tab:highest similarity}).

\begin{table}[h]
  \caption{File with the highest similarity}
  \label{tab:highest similarity}
  \centering
  \begin{tabular}{cclrrp{6cm}}
    \toprule
    index & annotator & filename & onset & offset & text \\
    \midrule
%    4228 & 1 & 203149.mp3 & 0.764318 & 1.250702 & Cows and bulls calling and mooing \\
    4228 & 1 & 203149.mp3 & 0.764318 & \dots & Cows and bulls calling and mooing \\
%    \vdots & \vdots & \vdots & \vdots & \vdots & \vdots \\
%    20575 & 1 & 203149.mp3 & 1.273863 & 1.737087 & Cows and bulls calling \\
%    26460 & 1 & 203149.mp3 & 1.760248 & 2.431921 & Cows and bulls calling and mooing \\
%    864 & 1 & 203149.mp3 & 3.312045 & 5.025971 & cows and bulls calling and mooing \\
%    31889 & 1 & 203149.mp3 & 6.647251 & 14.753655 & Cows and bulls calling and mooing \\
%    30194 & 1 & 203149.mp3 & 21.655679 & 23.045349 & cows and bulls calling and mooing \\
    30194 & 1 & 203149.mp3 & \dots & 23.045349 & cows and bulls calling and mooing \\
    21454 & 2 & 203149.mp3 & 0.064283 & 24.191995 & cows and bulls mooing \\
%    13107 & 1 & 203149.mp3 & 0.046322 & 0.602190 & birds singing in the country side \\
    13107 & 1 & 203149.mp3 & 0.046322 & \dots & birds singing in the country side \\
%    \vdots & \vdots & \vdots & \vdots & \vdots & \vdots \\
%    17270 & 1 & 203149.mp3 & 1.273863 & 1.737087 & Birds singing \\
%    5080 & 1 & 203149.mp3 & 1.876054 & 3.335206 & Birds singing \\
%    21397 & 1 & 203149.mp3 & 4.145847 & 7.434731 & Birds singing \\
%    33555 & 1 & 203149.mp3 & 14.915783 & 17.834089 & birds singing \\
%    13834 & 1 & 203149.mp3 & 19.131114 & 20.173365 & Birds singiing \\
%    20461 & 1 & 203149.mp3 & 23.184316 & 24.133923 & Birds singing \\
    20461 & 1 & 203149.mp3 & \dots & 24.133923 & Birds singing \\
    29140 & 2 & 203149.mp3 & 0.064283 & 24.191995 & birds singing \\
    \bottomrule
  \end{tabular}
\end{table}


\subsection{To what extent do the annotations rely on or deviate from keywords and textual descriptions in the audio’s metadata?}
\label{sec:Case Study:b}

In both cases, the metadata match the text field very well (Table~\ref{tab:metadata match}).

\begin{table}[h]
  \caption{File with the largest difference}
  \label{tab:metadata match}
  \centering
  \begin{tabular}{clp{4cm}p{8.5cm}}
    \toprule
    index & filename & title & keywords \\
    \midrule
    5182 & 568273.mp3 & spectral violin drone processed through granulation and reverb & spectral, tonal, granulation, horror, drone, dark, avant-garde, ambient, violin, soundscape, experimental \\
    6104 & 203149.mp3 & End of the afternoon in a field, in Nebraska & CD130519T018, felix, cows, singing, birds, end, usa, field, call, bird, evening, countryside, bulls, moo, bull, sing, mooing, blume, calls, cow, fields, calling, nebraska, afternoon \\
    \bottomrule
  \end{tabular}
\end{table}

\subsection{Was the temporal and text annotations done according to the task description?}
\label{sec:Case Study:c}

Yes, the annotations were made according to the task description. 
However, in the different case the quality is better. 
The similar case was greatly simplified.


