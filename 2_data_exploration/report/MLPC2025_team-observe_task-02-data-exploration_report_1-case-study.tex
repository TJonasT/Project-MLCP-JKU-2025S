
\section{Case Study}
\label{sec:Case Study}

To find 2 interesting records that were edited by multiple annotators, 
we first looked in ``metadata.csv'' to see which files had more than one annotator. 
This resulted in a list of 149 files. \\
We then looked at the ``metadata\_title\_embeddings.npz'' and the ``metadata\_keywords\_embeddings.npz'' in order 
to be able to draw some conclusions. 
At the same time, we also checked the standardization of the embedding. 

The approach was that files with very clear embedding (target: 1.0 in one place) lead to very clear annotations. 
However, this did not produce any useful results. \\

An important assumption here is the correctness and accuracy of the titles and keywords. \\


The next approach used the file ``annotations.csv'' and the corresponding ``annotations\_text\_embeddings.npz''. 
All annotations with more than one annotator were searched for. 
This resulted in a list of 731 annotations and 1468 annotations in total. There were 6 files with 3 annotations each. \\

From the associated text embeddings, we compared all embeddings of an annotator and an annotation with all other annotators by distance and subsequently removed 2 files.
The file with the largest deviation is '568273.mp3' and the file with the smallest deviation is '203149.mp3'.


\subsection{Identify similarities or differences between temporal and textual annotations from different annotators.}
\label{sec:Case Study:a}



\subsection{To what extent do the annotations rely on or deviate from keywords and textual descriptions in the audio’s metadata?}
\label{sec:Case Study:b}



\subsection{Was the temporal and text annotations done according to the task description?}
\label{sec:Case Study:c}




